\documentclass{article}
\usepackage[utf8]{inputenc}
\usepackage{multirow}
\title{PROBLEM SET 2: Predicate logic.}
\author{Pablo Cuesta Sierra}

\begin{document}

\maketitle

\section*{EXERCISE 1.}

	Considering the ontology:

	Constants: Thelma, Louise (people)
    
	Variables: p (people), x (objects)
    
	Predicates: \\
\begin{tabular}{|l|l|l|}
\hline
Name   & Arity & Description (including the type of arguments)   \\ \hline
Likes  & 2     &  Likes(p,x) evaluates to “True” if and only if p
likes the type of food x. \\ \hline
Cooked & 1     & Cooked(x) evaluates to “True” if and only if x
is cooked (not raw).  \\ \hline
Food   & 1     & Food(x) evaluates to “True” if and only if x is
a type of food.\\ \hline
Fish   & 1     & Fish(x) evaluates to “True” if and only if x is
a type of fish. \\ \hline
\end{tabular}\\
\\[\baselineskip]
Now we write in predicate logic the following statements:\\

I. "There are some types of food that Thelma dislikes but Louise likes":
$$\exists x [Food(x)\land \neg Likes(Thelma,x) \land Likes(Louise,x)]$$

II. “Louise dislikes all types of food that Thelma likes (and possibly
others)”:
$$\forall x[(Food(x)\land Likes(Thelma,x))\Longrightarrow \neg Likes(Louise,x)]$$

III. “Thelma likes all types of food except raw fish (which she dislikes)”:
$$\forall x\{[(Food (x) \Longrightarrow Likes(Thelma,x)] \iff [\neg Fish(x)\lor Cooked(x)]\}$$\\[\baselineskip]
\section*{EXERCISE 2.}
Ontology: \\
\begin{tabular}{|l|l|l|}
\hline
                            & Symbol                & Interpretation                      \\ \hline
Constant                    & $0$                     & Integer value of 0                  \\ \hline
Variables                   & $x$, $y$, $m$ & Integers                            \\ \hline
\multirow{2}{*}{Predicates} & $Equal(x,y)$            & True if and only if $x=y$            \\ \cline{2-3} 
                            & $NeutralSum(x)$         & True if and only if $x$ is the neutral element of the sum \\ \hline
\multirow{2}{*}{Functions}  & $Sum(x,y)$              & The result of $x+y$                   \\ \cline{2-3} 
                            & $Negative(x)$           & $-x$                                  \\ \hline
\end{tabular}\\ \\[\baselineskip]
NOTE: subtracting one element is  the same as adding its negative value, therefore $(x-y)$ would be $Sum(x,Negative(y)).$
\\[\baselineskip]
Formulate the following statements as WFFs in predicate logic.\\

a) "The result of subtracting two integers is zero if and only if these integers are equal":
$$\forall x,y [Equal(Sum(x,Negative(y)),0)\iff Equal(x,y)]$$

b) "An integer number is the neutral element with respect to the sum if and only if when subtracted from any integer the result is equal to this second
integer":
$$\forall x\{NeutralSum(x)\iff \forall y[Equal(Sum(y,Negative(x)),y)]\}$$

c) "For any integer there is another integer such that the result of subtracting
the second integer from the first one yields the neutral element":
$$\forall x\{NeutralSum(x) \Longrightarrow \forall y\exists m [Equal(Sum(y,Negtive(m)),x)]\}$$

d) "The neutral element with respect to the sum is unique and equal to zero":
$$\forall x (NeutralSum(x) \iff Equal (x,0))$$\\[\baselineskip]\\[\baselinskip]\\[\baselinskip]\\[\baselinskip]\\[\baselinskip]\\[\baselinskip]\\[\baselinskip]
\section*{EXERCISE 3.}
Let us consider a series of elections in which only three voters
can cast votes according to their preferences.\\
\begin{tabbing}
- Variables:  \= \>x, y, z, ... (candidates)\\ \>p, q, r, ... (voters)\\
- Predicates:  \= 
\> - $P^3$ “Prefers”: P(p,x,y) evaluates to True if voter p prefers
can-\\ \>didate x to candidate y, False otherwise.\\
\> - $B^2$ “Beats”: B(x,y) evaluates to True if candidate x beats
can-\\ \>didate y in a two-candidate election, False otherwise.
\end{tabbing}
Translate the following knowledge base into well-formed formulas in predicate
logic. The predicate “Equal” ($E^2$) can be used if needed.

(a) ``Predicate “Beats” is antisymmetric: If a candidate beats another one, the
second one does not beat the first one":
$$\forall x, y\ [B(x,y)\iff \neg B(y,x)]$$

(b) ``Predicate “Prefers” is transitive: If a voter prefers a candidate to another one, and also prefers this second candidate to a third one, she prefers the first candidate to the third one."
$$\forall x,y,z,p \ \{[P(p,x,y)\land P(p,y,z)] \Longrightarrow P(p,x,z)\}$$

(c) ``In a two-candidate election, if at least two different voters prefer one
candidate to another one, the first one beats the second one."
$$\forall x,y\{\exists p,q [P(p,x,y)\land P(q,x,y)\land \neg Equal(p,q)]\Longrightarrow B(x,y)\}$$

(d) ``Predicate “Beats” is not transitive."
$$\forall x,y,z \ \{[B(x,y)\land B(y,z)]\Longrightarrow [B(x,z)\lor B(z,x)]\}$$ (In this case, we do not contemplate the possibility of the two candidates x and z having the exact same amount of votes, so either x beats z or vice versa.)



\end{document}
